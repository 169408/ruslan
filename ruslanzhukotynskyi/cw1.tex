\documentclass[12pt, letterpaper, titlepage]{article}
\usepackage[left=3.5cm, right=3.5cm, top=2.5cm, bottom=2.5cm]{geometry}
\usepackage[MeX]{polski}
\usepackage[utf8]{inputenc}
\usepackage{graphicx}
\usepackage{enumerate}
\usepackage{amsmath}
\usepackage{amssymb}
\title{Pierwszy dokument LaTex}
\author{Ruslan Zhukotynskyi}
\date{18 Paździrnika 2022}
\begin{document}
\maketitle
\section{Sport to zdrowie}
\subsection{Dlaczego się mówi sport to zdrowie?}
\begin{enumerate}
\item Z pewnością każdy, kto uprawia jakiś rodzaj sportu, dostrzega jego pozytywny wpływ na swoje samopoczucie. Jest to jeden z podstawowych czynników, wpływających na zdrowe funkcjonowanie organizmu, dobrą kondycję czy zgrabną sylwetkę. W czym konkretnie może pomóc regularna aktywność fizyczna?
\item Uprawianie sportu pomaga zwalczyć otyłość. To bardzo istotny czynnik, gdyż zbyt duża masa ciała przyczynia się do chorób stawów, nerek, cukrzycy, a także zawałów serca. Regularne ćwiczenia w ogóle zmniejszają ryzyko występowania przypadłości kardiologicznych. Przeprowadzone badania wykazały, że osoby aktywne fizycznie są dwa razy mniej podatne na choroby serca, niż te, które nie uprawiają sportu.
\item Ponadto ćwiczenia są podstawą w profilaktyce związanej z nadciśnieniem tętniczym. Zachowywanie aktywności fizycznej sprzyja także wzmacnianiu mięśni, ścięgien i wiązadeł. To z kolei sprawia, że jesteśmy bardziej sprawni i mniej narażeni na dolegliwości związane z tymi częściami ciała. Uprawianie sporu minimalizuje też ryzyko zachorowania na cukrzycę typu 2, i to o ponad 30 procent. Natomiast tym, którzy już chorują na cukrzycę, aktywność fizyczna pomaga w wyrównaniu stężenia cukru we krwi. Niezwykle ciekawą rzeczą jest fakt, że ćwiczenia mogą zmniejszyć ryzyko zachorowań na nowotwory piersi, płuc lub jelita grubego. Jest to z pewnością silna motywacja, by regularnie zażywać ruchu – zwłaszcza na świeżym powietrzu.
\end{enumerate}
\section{Sport to zdrowie – Naturalny suplement usprawniający funkcjonowanie umysłu.}
Sport jest jednym z lepszych środków prewencyjnych, hamuje rozwój chorób otępiennych, zwłaszcza Alzheimera. Warto zatem zachęcać seniorów do aktywności fizycznej. Nic tak bardzo nie pogarsza sprawności umysłowej starszych ludzi jak bierne i samotne przebywanie w czterech ścianach swojego domu. Myślę, że dobrym rozwiązaniem byłoby tu organizowanie zajęć kierowanych specjalnie do osób starszych. Zajęć poprzedzonych odpowiednią kampanią społeczną zachęcającą do takiej aktywności, np. poprzez pokazywanie przykładów osób starszych, aktywnie spędzających czas.
\newpage
\section{Wystarczy już 30 minut ćwiczeń. I regularność.}
Badania pokazują, że regularne uprawianie sportu z umiarkowaną lub dużą intensywnością (ale nie przetrenowywanie się bo wtedy efekt może być odwrotny) wpływa na poprawę funkcjonowania poznawczego (m. in. koncentrację, pamięć i zdolność uczenia się). Chroni też przed pogarszaniem się pamięci i spowalnia procesy demencyjne u osób starszych (i nie tylko).
\subsection{Sport}
Sport jest jednym z lepszych środków prewencyjnych, hamuje rozwój chorób otępiennych, zwłaszcza Alzheimera. Warto zatem zachęcać seniorów do aktywności fizycznej. Nic tak bardzo nie pogarsza sprawności umysłowej starszych ludzi jak bierne i samotne przebywanie w czterech ścianach swojego domu. Myślę, że dobrym rozwiązaniem byłoby tu organizowanie zajęć kierowanych specjalnie do osób starszych. Zajęć poprzedzonych odpowiednią kampanią społeczną zachęcającą do takiej aktywności, np. poprzez pokazywanie przykładów osób starszych, aktywnie spędzających czas.
\newpage
\section{tekst}
Hello World!
\subsection{tekst}
We are from Ukraine!
\subsubsection{tekst}
We dont know what we are doing right here and now!
\end{document}